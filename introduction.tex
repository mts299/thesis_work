\documentclass{article}
\usepackage[utf8]{inputenc}
\usepackage{authblk}
\usepackage{amsmath}

\title{Designing error control circuits for 4-qubit quantum computing}
\author[1]{Marina Schmidt}
\author[1]{Dr. R. Spiteri}
\affil[1]{Department of Computer Science, University of Saskatchewan}
\affil[ ]{Email: \textit {mts299@mail.usask.ca}}
\affil[ ]{NSID: mts299}

\begin{document}

\maketitle

\section{Introduction}
 Quantum computers hold a promising advantage to solving hard problems, such as prime factorization, within a sufficient amount of time. This is because of their ability to harness the quantum states of subatomic particles (e.g. photons and electrons) to represent a quantum bit (qubit). Similar to a classical bit, a qubit can represent zero or one state based on their subatomic properties (e.g, spin); however, a qubit can also be in both states when not being measured because of quantum superposition. Because of this quantum property, when a qubit is added to the system it can interact with the other qubits to create superposition states that will increase the amount of information provided to the system. For example, a two qubit system can be represented as: 
\begin{align*}
&\delta |11\big\rangle \\ 
&\gamma |01 + 10 \big\rangle \\
&\beta  |01 - 10 \big\rangle \\
&\alpha |00 \big\rangle \\
&\end{align*}
Therefore, four bits of information ($\alpha,\beta,\gamma$ and $\delta$) are required to describe the system, whereas a two bit classical system only takes two bits of information to describe the system. With this property of superposition, entanglement of qubits can allow an exponential of two to represent the information. However, when the system state is measured the qubit system has to fall into one of the basis states ($|11\big\rangle$ or $|00 \big\rangle$). To determine the superposition state the qubit system coherence pulse can be measured by using a series of logical operations. 

The ability of the qubit to interact with other qubit is also problematic because it can interact with other environmental subatomic particles from which it is impossible to isolate a qubit system. When this undesired interaction occurs, decoherence or loss of information causes errors in the information read from the system. Decoherence creates a dampening affect on the resulting pulse read from the qubit system when this dampening hits a flat line, decoherence time, all information is lost from the system. To prevent the major loss of information from decoherence, error correction operations are applied to obtain the resulting information from the system. One constraint on this is the error correction time needs to be less than decoherence time. Another constraint on error correction in a qubit system is it cannot use redundancy similar to classical bit error correction because of the no-cloning theorem. This theorem states that one cannot clone a qubit system because of uncertantity in the superposition states. Therefore another method of error correction needs to be designed to handle the quantum noise and decoherence of the system. 

To obtain fault tolerance in a qubit system, several models of circuit operations to be applied to the qubit system have been designed. These models use circuit constants that manipulate frequencies in the qubit system to obtain the corrected readings of the qubit system. The overall reading can then be used to obtain an intrinsic fidelity. The intrinsic fidelity is without circuit noise and it is used to theoretically determine the reliability of the circuit gate design. Therefore by optimizing the gate constants to obtain the desired fidelity, a designed circuit gate for error correction can be used to create a fault tolerant qubit system. 

In this poster the Tofolli single shot gate is optimized to obtain an intrinsic fidelity of $99.99\%$. It promises enough robustness in the error correction gate to guarantee fault tolerance without losing too much gate time so as to stay lower than the decoherence time. The four qubit system is examined in this poster because it lowest enough system to create encoding codes for encryption and logical operations in quantum machine that then can be taken advantage of in five qubit system for the overall decryption process for security applications. By optimizing the tofolli single shot security for a four qubit system with an intrinsic fidelity of $99.99\%$, this circuit design can be used for further research into quantum encryption.              
\end{document}

