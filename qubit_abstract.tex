\documentclass{article}
\usepackage[utf8]{inputenc}
\usepackage{authblk}

\title{Designing error control circuits for 4-qubit quantum computing}
\author[1]{Marina Schmidt}
\author[1]{Dr. R. Spiteri}
\affil[1]{Department of Computer Science, University of Saskatchewan}
\affil[ ]{Email: \textit {mts299@mail.usask.ca}}
\affil[ ]{NSID: mts299}

\begin{document}

\maketitle

\begin{abstract}

Quantum computing offers the promise of a spectral leap in performance compared to transistor-based computers. The performance increase is achieved by the ability of a quantum bit to represent a relatively large number of states. Accordingly, less time is needed to read and write a given bit string when compared to classical computer, and a larger architecture size can be represented. However, with the promise of increased performance comes the obligation of increased error checking and
    correction of qubits in order to guarantee sufficient robustness to fault tolerance. To guarantee robust fault tolerance in a practical quantum computing system, an error control with an intrinsic (theoretical) fidelity of $99.99\%$. 
%
This fidelity value is required because of the electric current pulse used to control a qubit system in a controlled not gate, with additive noise known as the gate fidelity, resulting fault tolerance guarantee such that all common errors to occur will be
    corrected. 
% previous sentence needs work
We have achieved a circuit design with intrinsic fidelity of $99.9908\%$ at a processing time of 75 nanoseconds. 
% "processing time" needs explaining
The search took 3 months of computer time to find a feasible solution with the help of Matlab's Global Optimization Toolbox. This finding will allow experimentalists to build such a device and tackle even harder problems, applied to areas such as networking and security, than the current three-qubit systems are capable of handling.

\end{abstract}

\end{document}
