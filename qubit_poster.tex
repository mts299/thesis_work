%%%%%%%%%%%%%%%%%%%%%%%%%%%%%%%%%%%%%%%%%%%%%%%%%%%%%%%%%%%%%%%%%%%%%%%%
%%%                       May 2005                                   %%%
%%%	                                    				               %%%
%%%                a0 poster latex document		                     %%%
%%%       based on template supplied by Andreas Jung                 %%%
%%%           of the University of Regensburg.                       %%%
%%%             For more info, see README file                       %%%
%%%					                                   	               %%%
%%%                   Andrea Taroni, UCL                             %%%
%%% http://www.ucl.ac.uk/~uccaata/work/poster/poster.html            %%%
%%%                                                                  %%%
%%% NB. To print poster in a4 format, place                          %%%
%%%  \documentclass[landscape,a0b,final,a4resizeable]{a0poster}      %%%
%%% in your first line of your tex-file after compiling,             %%%
%%% run the program:  psresize -W595 -H842 poster.ps poster-a4.ps    %%%
%%% Then replace in "poster-a4.ps" the line:                         %%%
%%%  %%BoundingBox: 0 0 2594 3402  with  %%BoundingBox: 0 0 595 842  %%%
%%%%%%%%%%%%%%%%%%%%%%%%%%%%%%%%%%%%%%%%%%%%%%%%%%%%%%%%%%%%%%%%%%%%%%%%

%\documentclass[landscape,a0b,final,a4resizeable]{a0poster}
\documentclass[landscape,a0b,final]{a0poster}
%%% Option "a4resizeable" makes it possible ot resize the
%   poster by the command: psresize -pa4 poster.ps poster-a4.ps
%   For final printing, remove option "a4resizeable" !!

\usepackage{epsfig}
\usepackage{multicol}
\usepackage{multirow}
\usepackage{amsmath}
\usepackage{times}
%\usepackage{multirow}
%\usepackage{amsbsy}
\usepackage{amsfonts}
\usepackage{graphics}
\usepackage{graphicx, alltt, setspace}
%\usepackage{keyval}
\usepackage{caption2}
\usepackage{pstricks,pst-grad}
\graphicspath{{figures/}}
\usepackage[it]{subfigure}
\usepackage{subfigure}
\usepackage{url}
%%%%%%%%%%%%%%%%%%%%%%%%%%%%%%%%%%%%%%%%%%%
% Definition of some variables and colours
%\renewcommand{\rho}{\varrho}
%\renewcommand{\phi}{\varphi}
\setlength{\columnsep}{3cm}
\setlength{\columnseprule}{2mm}
\setlength{\parindent}{0.0cm}
% Some commands that may be useful from my cache of self-defined commands
 \newcommand{\bm}[1]{\mathbf{#1}}
\newcommand{\ol}{\overline}
\newcommand{\df}[2]{\frac{d #1}{d #2}}
\newcommand{\mV}{\mbox{\rm mV}}
\newcommand{\lb}{\left}
\newcommand{\rb}{\right}
\newcommand{\atol}{\texttt{atol}}
\newcommand{\rtol}{\texttt{rtol}}
\newcommand{\dt}{\Delta t}
\newcommand{\Na}{\mathrm{Na}}
\newcommand{\Ca}{\mathrm{Ca}}
\newcommand{\K}{\mathrm{K}}
\newcommand{\beq}{\begin{equation}}
\newcommand{\eeq}{\end{equation}}
\newcommand{\beqa}{\begin{eqnarray*}}
\newcommand{\eeqa}{\end{eqnarray*}}
\newcommand{\reword}[1]{  \textbf{#1} }
\newcommand{\ff}{\bm f}
\newcommand{\yy}{\bm y}
\newcommand{\DtMax}{\Delta t_{\text{max}}}
\newtheorem{proposition}{Proposition}[section]

\newtheorem{definition}[proposition]{Definition}
\newtheorem{remark}[proposition]{Remark}
\newtheorem{thm}{Theorem}[section]
\newtheorem{cor}[thm]{Corollary}
\newtheorem{lem}[thm]{Lemma}
\newtheorem{prop}[thm]{Proposition}
\newcommand{\pderiv}[2]{\frac{\partial #1}{\partial #2}}

	
%%%%%%%%%%%%%%%%%%%%%%%%%%%%%%%%%%%%%%%%%%%%%%%%%%%%
%%%               Background                     %%%
%%%%%%%%%%%%%%%%%%%%%%%%%%%%%%%%%%%%%%%%%%%%%%%%%%%%

\newcommand{\background}[3]{
  \newrgbcolor{cgradbegin}{#1}
  \newrgbcolor{cgradend}{#2}

  % Frame around poster
  \psframe[fillstyle=gradient,gradend=cgradend,
  gradbegin=cgradbegin,gradmidpoint=#3](0.,0.)(1.\textwidth,-1.\textheight)
}

%%%%%%%%%%%%%%%%%%%%%%%%%%%%%%%%%%%%%%%%%%%%%%%%%%%%
%%%                Poster                        %%%
%%%%%%%%%%%%%%%%%%%%%%%%%%%%%%%%%%%%%%%%%%%%%%%%%%%%

\newenvironment{poster}{
  \begin{center}
  \begin{minipage}[c]{0.98\textwidth}
}{
  \end{minipage} 
  \end{center}
}



%%%%%%%%%%%%%%%%%%%%%%%%%%%%%%%%%%%%%%%%%%%%%%%%%%%%
%%%                pcolumn                       %%%
%%%%%%%%%%%%%%%%%%%%%%%%%%%%%%%%%%%%%%%%%%%%%%%%%%%%

\newenvironment{pcolumn}[1]{
  \begin{minipage}{#1\textwidth}
  \begin{center}
}{
  \end{center}
  \end{minipage}
}



%%%%%%%%%%%%%%%%%%%%%%%%%%%%%%%%%%%%%%%%%%%%%%%%%%%%
%%%                pbox                          %%%
%%%%%%%%%%%%%%%%%%%%%%%%%%%%%%%%%%%%%%%%%%%%%%%%%%%%

%Test this one first
\newrgbcolor{lcolor}{0. 0. 0.80}
\newrgbcolor{gcolor1}{1. 1. 1.}
\newrgbcolor{gcolor2}{.80 .80 1.}
\newcommand{\pbox}[4]{
\psshadowbox[#3]{
\begin{minipage}[t][#2][t]{#1}
#4
\end{minipage}
}}



%%%%%%%%%%%%%%%%%%%%%%%%%%%%%%%%%%%%%%%%%%%%%%%%%%%%
%%%                myfig                         %%%
%%%%%%%%%%%%%%%%%%%%%%%%%%%%%%%%%%%%%%%%%%%%%%%%%%%%
% \myfig - replacement for \figure
% necessary, since in multicol-environment 
% \figure won't work

\newcommand{\myfig}[3][0]{
\begin{center}
  \vspace{0.5cm} %{1.5cm}
  \includegraphics[width=#3\hsize,angle=#1]{#2}
  \nobreak\medskip
\end{center}}

\newcommand{\myfigsub}[2]{
\begin{center}
  \vspace{0.5cm} %{1.5cm}
  %\includegraphics{#1}
  %\includegraphics{#2}
  \includegraphics[height=125mm,bb=25 169 579
    609]{#1}
\includegraphics[height=125mm,bb=25 169 579
    609]{#2}
    %\includegraphics[height=100mm]{#1}
    %\includegraphics[height=100mm]{#2}
  \nobreak\medskip
\end{center}}



%%%%%%%%%%%%%%%%%%%%%%%%%%%%%%%%%%%%%%%%%%%%%%%%%%%%
%%%                mycaption                     %%%
%%%%%%%%%%%%%%%%%%%%%%%%%%%%%%%%%%%%%%%%%%%%%%%%%%%%
% \mycaption - replacement for \caption
% necessary, since in multicol-environment \figure and
% therefore \caption won't work

%\newcounter{figure}
\setcounter{figure}{1}
\newcommand{\mycaption}[1]{
  %\vspace{0.5cm}
  \begin{quote}
    {{\sc Figure} \arabic{figure}: #1}
  \end{quote}
  \vspace{0.3cm}
  \stepcounter{figure}
}

%%%%%%%%%%%%%%%%%%%%%%%%%%%%%%%%%%%%%%%%%%%%%%%%%%%%
%%%                myTableCaption                %%%
%%%%%%%%%%%%%%%%%%%%%%%%%%%%%%%%%%%%%%%%%%%%%%%%%%%%
% \myTableCaption - replacement for \caption
% necessary, since in multicol-environment \figure and
% therefore \caption won't work

\setcounter{table}{1}
\newcommand{\myTableCaption}[1]{
  \vspace{0.5cm}
  \begin{quote}
    \begin{center}
      {{\sc Table} \arabic{table}: #1}
    \end{center}
  \end{quote}
  %\vspace{0.5cm}
  \stepcounter{table}
}

%%%%%%%%%%%%%%%%%%%%%%%%%%%%%%%%%%%%%%%%%%%%%%%%%%%%%%%%%%%%%%%%%%%%%%
%%% Beginning of Document
%%%%%%%%%%%%%%%%%%%%%%%%%%%%%%%%%%%%%%%%%%%%%%%%%%%%%%%%%%%%%%%%%%%%%%

\begin{document}

%background color
%\background{.05 .15 0.3}{.05 .15 0.3}{0.0}
%\background{0.0 .42 .24} {0.0 .42 .24} {0.0}
%\background{255 255 255}
\newrgbcolor{bluegreen}{.1 .4 .45}
\newrgbcolor{lightblue}{.1 .4 .45}
\newrgbcolor{white}{1. 1. 1.}
\newrgbcolor{yellow}{1. 0.9 0.4}
\newrgbcolor{UofSgreen}{1. 0.9 0.4}

\newrgbcolor{greyBlack}{43 40 40}

\vspace*{2cm}

%Line colour
%\newrgbcolor{lightblue}{0. 0. 0.}
%Used for gradient
%\newrgbcolor{white}{1. 1. 1.}
%\newrgbcolor{whiteblue}{.80 .80 1.}
%\newrgbcolor{UofSgreen}{.125 .5 .25}

%Gradient
%\newrgbcolor{lightUofSgreen}{1. .9 0.4}
\newrgbcolor{lightUofSgreen}{.356 .647 .0745}
\newrgbcolor{lightUofSgreen2}{.029 .49 .039}
\begin{poster}

%%%%%%%%%%%%%%%%%%%%%
%%% Header
%%%%%%%%%%%%%%%%%%%%%
\begin{center}
\begin{pcolumn}{0.98}

%\pbox{0.95\textwidth}{}{linewidth=1mm,framearc=0.1,linecolor=black,fillstyle=gradient,gradangle=0,gradbegin=lightUofSgreen,gradend=lightUofSgreen2,gradmidpoint=1.0,framesep=1em}{

\pbox{0.98\textwidth}{}{linewidth=2mm,framearc=0.1,linecolor=black,fillstyle=gradient,gradangle=0,gradbegin=white,gradend=white,gradmidpoint=1.0,framesep=1em}{
%%% UofS Logo
\begin{minipage}[c][9cm][c]{0.15\textwidth}
  \begin{center}
    \includegraphics[width=15cm,angle=0]{UofS}
  \end{center}
\end{minipage}
%%% Title
\begin{minipage}[c][10cm][c]{0.7\textwidth}
\vspace*{1cm}
  \begin{center}
    {\sc \textcolor{black}{\Huge The secrets to the success of the Rush--Larsen method and its  generalizations}}\\[15mm]
    {\Large  Megan E. Marsh, Saeed Torabi Ziaratgahi, and Raymond J. Spiteri\\[7.5mm]
    Simulation Research Laboratory, Department Computer Science, University of Saskatchewan \\[7.5mm]
    email: \ttfamily spiteri@cs.usask.ca}
  \end{center}
\end{minipage}
%%% NSERC Logo
\begin{minipage}[c][9cm][c]{0.15\textwidth}
  \begin{center}
    \includegraphics[width=15cm,angle=0]{NSERC}
  \end{center}
\end{minipage}

}
\end{pcolumn}
\end{center}

\vspace*{.5cm}


%%%%%%%%%%%%%%%%%%%%%
%%% Content
%%%%%%%%%%%%%%%%%%%%%
\begin{center}
\begin{pcolumn}{0.24}
\pbox{0.9\textwidth}{66cm}{linewidth=2mm,framearc=0.1,linecolor=black,fillstyle=gradient,gradangle=0,gradbegin=white,gradend=white,gradmidpoint=1.0,framesep=1em}{

%%% Introduction
\begin{center}\pbox{0.8\textwidth}{}{linewidth=1mm,framearc=0.1,linecolor=black,fillstyle=gradient,gradangle=0,gradbegin=lightUofSgreen,gradend=lightUofSgreen2,gradmidpoint=1.0,framesep=1em}{\begin{center}\sc \textcolor{white}{\Large{Introduction}}\end{center}}\end{center}
\vspace{0.5cm}

  One of the most popular methods for solving the ordinary
  differential equations (ODEs) that describe the dynamic behaviour of
  myocardial cell models is known as the Rush--Larsen (RL) method. Its
  popularity stems from its improved stability over integrators such
  as the forward Euler (FE) method along with its easy
  implementation. The RL method partitions the ODEs into two sets: one
  for the gating variables, which are treated by an exponential
  integrator, and another for the remaining equations, which are
  treated by the FE method. The success of the RL method can be
  understood in terms of its relatively good stability when treating
  the gating variables. However, this feature would not be expected to
  be of benefit on cell models for which the stiffness is not captured
  by the gating equations. We demonstrate that this is indeed the case
  on a number of stiff cell models. We further propose a new
  partitioned method based on the combination of a first-order
  generalization of the RL method with the FE method. This new method
  leads to simulations of stiff cell models that are often one or two
  orders of magnitude faster than the original RL method.


%%% Myocardial cell models
\begin{center}\pbox{0.8\textwidth}{}{linewidth=1mm,framearc=0.1,linecolor=black,fillstyle=gradient,gradangle=0,gradbegin=lightUofSgreen,gradend=lightUofSgreen2,gradmidpoint=1.0,framesep=1em}{\begin{center}\sc \textcolor{white}{\Large{Myocardial cell models}}\end{center}}\end{center}
\vspace{0.5cm}

A wide range of models has been developed to describe the electrical
currents in various single heart cells, e.g., atrial cells,
ventricular cells, human cells, rat cells, etc. Most can be formulated
as an initial-value problem (IVP) for a system of ODEs of the form
\begin{equation}
\frac{d{\bf y}}{dt} = {\bf f}(t,{\bf y}), \qquad
 {\bf y}(t_0)=\bf{y_0}.
  \label{eq:IVP}
\end{equation}

The component variables of the vector ${\bf y}$ are dependent on the
cell model but they typically include the transmembrane potential, a
number of gating variables, and a set of ionic concentrations and
can be written as
\begin{subequations}
\label{eq:cellmodel}
\begin{align}
\label{eq:dVmdt}
\frac{dV_m}{dt}&=-\frac{1}{C_m} \sum\limits_{i=1}^{n_{ion}}
I_i(V_m,{\bf m},{\bf c},t),\\
\label{eq:dcjdt}
\frac{dc_j}{dt}&=g_j(c_j,{\bf m},V_m,t), &j&=1,2,...,n_c,\\
\label{eq:dmkdt}
\frac{dm_k}{dt}&= \alpha_k (1 -m_k)-\beta_k m_k, &k&=1,2,...,n_m.
\end{align}
\end{subequations}

Equation (\ref{eq:dVmdt}) describes the evolution of the transmembrane
potential $V_m$, equation (\ref{eq:dcjdt}) describes
the dynamic variations in intracellular ionic
concentrations, and equation (\ref{eq:dmkdt}) describes the opening and
closing of ion channels in the cell membrane. 

An important consideration in the efficient numerical solution of
differential equations is the concept of {\em stiffness}. In this study, 
an IVP (\ref{eq:IVP}) is considered to be stiff on a time interval with 
respect to a given numerical method and error tolerance when stability 
requirements force the numerical method to take smaller step sizes than 
those dictated by accuracy requirements.  In this study we consider the 
numerical solution of 37 different myocardial cell models. 
The cell models considered in this study range from non-stiff to
moderately stiff to stiff for typical accuracy requirements.

Related to the stiffness of an IVP (\ref{eq:IVP}) are the
eigenvalues of the Jacobian matrix, $\bf{J}=\frac{\partial {\bf
 f}}{\partial y} (\it{t},{\bf y})$, evaluated over time. The
magnitude and nature of these eigenvalues can provide information as to the degree
of stiffness present in an IVP at a given time. A stiff IVP typically
has eigenvalues $\lambda$ with large negative real parts on some time
interval. Such eigenvalues force the time step $\Delta t$ to be small
so that $\lambda\Delta t$ is within the stability region of the
numerical method. 



}
\end{pcolumn}
																%%%%%%%%%%%%%%
																% 2nd column %
																%%%%%%%%%%%%%%
\begin{pcolumn}{0.24}
\pbox{0.9\textwidth}{66cm}{linewidth=2mm,framearc=0.1,linecolor=black,fillstyle=gradient,gradangle=0,gradbegin=white,gradend=white,gradmidpoint=1.0,framesep=1em}{
\vspace{0.2cm}


%%% Basic Numerical Methods
\begin{center}\pbox{0.8\textwidth}{}{linewidth=1mm,framearc=0.1,linecolor=black,fillstyle=gradient,gradangle=0,gradbegin=lightUofSgreen,gradend=lightUofSgreen2,gradmidpoint=1.0,framesep=1em}{\begin{center}\sc \textcolor{white}{\Large{Basic Numerical Methods}}\end{center}}\end{center}
\vspace{0.4cm}


Given the IVP
\begin{equation}
\frac{d{\bf y}}{dt} = {\bf f}({\it t},{\bf y}), \qquad
 {\bf y}({\it {t_n}})={\bf {y_n}},
  \label{eq:cellmodelIVP}
\end{equation}
for $t_n<t<t_{n+1}$, where ${\bf y} \in \mathbb{R}^M$, ${\bf f} :
\mathbb{R} \times \mathbb{R}^M \rightarrow \mathbb{R}^M$, and $\Delta t_n =
t_{n+1} - t_n $, the FE method approximates (\ref{eq:cellmodelIVP}) by
\begin{equation}
\label{eq:FEformula}
{\bf y}_{n+1}={\bf y}_n + {\it {\Delta t_n}} \hspace{1mm} {\bf f}(t_n,{\bf y}_n).
\end{equation}

The RL method applies the FE method to the ODEs for non-gating
variables present in (\ref{eq:cellmodelIVP}) but uses a different
technique for the ODEs satisfied by gating variables. These ODEs have
the form (\ref{eq:dmkdt}) that, for a typical gating variable $y$, can
be reformulated as
\begin{equation}  
\frac{dy}{dt} = \frac{y_{\infty} - y}{\tau_y},
\label{eq:gatingODEs}
\end{equation}
where 
\begin{equation*}
y_{\infty} = \frac{\alpha_y}{\alpha_y + \beta_y}, \hspace{10mm}\tau_y = \frac{1}{\alpha_y + \beta_y}.
\end{equation*}
The RL method assumes the transmembrane potential $V_m$ is constant
over each step, allowing (\ref{eq:gatingODEs}) to be treated as a
linear ODE with an exact solution given by
\begin{equation}
y_n = y_{\infty} + (y_{n-1}
-y_{\infty}) e^{-\frac{\Delta t_n}{\tau_y}}.
\label{eq:RLformula}
\end{equation}

The GRL1 method decouples and linearizes the ODE system around a point
${\bf y}={\bf y}_n$ at time $t=t_n$ to obtain
\begin{equation}
   \frac{{d} y_{i}}{{d} t} = f_{i}({\bf y}_n) + \frac{\partial}{\partial y_{i}} f_{i} ({\bf y}_n)\left( y_{i} - y_{n,i} \right), \quad y_{i} (t_{n}) = y_{n,i},
\label{eq:decoupledODEs} 
\end{equation}
for $i=1, 2, \hdots, M$, where the subscript $i$ denotes component $i$ 
of a vector. The numerical solution ${\bf y}_{n+1}$ at time $t=t_{n+1}$ is obtained by
\begin{equation}
y_{n+1,i} = y_{n,i} + \frac{a}{b} \left( e^{ b ( \Delta t_{n})} - 1 \right), \quad  i=1, 2, \hdots, M.
\label{eq:GRL1mainformula}
\end{equation}

The RL and GRL1 methods treat the gating equations
(\ref{eq:dmkdt}) similarly. In other words, if $y_i$ is a
gating variable then (\ref{eq:GRL1mainformula}) reduces to
(\ref{eq:RLformula}). The key difference between the methods is in
their treatment of the non-gating variables: GRL1 applies an exponential
integrator based on local linearization to non-gating variables whereas
RL uses the FE method. A summary of the three basic numerical methods used for this study is
presented in the following table:

\vspace{0.5 cm} 
\begin{tabular}{|l|c|c|}\hline 
  \multirow{2}{*}{\bf{Method}} & \bf{Gating variables} & \bf{Non-gating variables}\\
  & \bf{(gating equations)} & \bf{(non-linear equations)}\\
  \hline
  \multirow{2}{*}{\bf{FE}} & \multirow{2}{*}{FE integrator (\ref{eq:FEformula})}
  & \multirow{2}{*}{FE integrator (\ref{eq:FEformula})}\\
  & &\\
  \hline
  \multirow{2}{*}{\bf{RL}} & \multirow{2}{*}{Exponential integrator (\ref{eq:GRL1mainformula})} &
  \multirow{2}{*}{FE integrator (\ref{eq:FEformula})}\\
  & &\\
  \hline
  \multirow{2}{*}{\bf{GRL1}} & \multirow{2}{*}{Exponential integrator (\ref{eq:GRL1mainformula})} & Local
  linearization (\ref{eq:decoupledODEs}) +\\
  &  & Exponential integrator (\ref{eq:GRL1mainformula})\\
  \hline
  \noalign{\smallskip}
\end{tabular}
\vspace{0.5 cm}
\\We note that the method
that is the least stable but computationally cheapest per step is the
FE method and the method that is the most stable but computationally
costliest per step is the GRL1 method. This tradeoff of stability for
computational cost per step is typical for numerical methods to solve
stiff IVPs. It is often the case that the increase in stable step size
more than offsets increase in computational cost per step, leading to
a less expensive computation (i.e., more efficient method) overall.

}
																%%%%%%%%%%%%%%
																% 3rd column %
																%%%%%%%%%%%%%%
\end{pcolumn}
\begin{pcolumn}{0.24}
\pbox{0.9\textwidth}{66cm}{linewidth=2mm,framearc=0.1,linecolor=black,fillstyle=gradient,gradangle=0,gradbegin=white,gradend=white,gradmidpoint=1.0,framesep=1em}{

%%% Partitioned Method
\begin{center}\pbox{0.8\textwidth}{}{linewidth=1mm,framearc=0.1,linecolor=black,fillstyle=gradient,gradangle=0,gradbegin=lightUofSgreen,gradend=lightUofSgreen2,gradmidpoint=1.0,framesep=1em}{\begin{center}\sc \textcolor{white}{\Large{Partitioned Method}}\end{center}}\end{center}
\vspace{0.4cm}
By analyzing the eigenvalues of the Jacobian matrix of cell models, 
it can be determined that the five stiffest cell models out of 37 
are Bondarenko et al. (2004), Jafri et al. (1998), Pandit et al. 
(2003), the Endocardial variant of Ten Tusscher et al. (2004),
and Winslow31. Eigenvalue analysis also determines that 
only a few of the ODEs are
responsible for the stiffness of the model. This provides a means by
which the system of ODEs can be partitioned into stiff and non-stiff
subsystems. In addition, this eigenvalue analysis reveals on which
sub-interval(s) of the entire interval of integration the IVP is
stiff. This permits a partitioning of the interval of integration into
stiff and non-stiff subintervals.

As example, we consider the stiff cell model of Pandit et
al. (2003). The plot of the real parts of the
eigenvalues of the Jacobian matrix of this model is given in the following picture:
\begin{center}
	\myfig{Pandit2003(zoom)}{1}
	\mycaption{Real parts of eigenvalues of Jacobian over time for the
    model of Pandit et al. (2003)}
\end{center}

The plot shows the ODEs that capture the
stiffness of the system are not stiff on the entire interval of
integration. From close examination of the eigenvalues, we find that
only two ($P_{C1}$ and $P_{O1}$) out of 26 ODEs from the model of Pandit et al. (2003)
are responsible for the stiffness of the models. We also identify
that the stiffness is approximately contained within the subintervals 
$[105,195]$.	

An important point to note is that the majority of stiff variables are not gating
variables. This means that most of the stiffness of stiff cell models is
not captured by gating variables. In the case of stiff
models for which the stiffness is not captured by gating variables, we
expect the RL method to perform less well because its step size can be
adversely impacted by stability restrictions imposed by the FE method
being applied to stiff non-gating equations. For such models, we
expect a method such as GRL1 that treats stiff non-gating equations
with an exponential integration method to outperform the RL
method. 

Furthermore we expect a combination of the GRL1 method and the
FE method that takes advantage of partitioning the ODE system and time
interval into stiff and non-stiff subsets to perform even more
effectively. Specifically we propose to use the FE method for the
entire ODE system on the non-stiff portion of the time domain and the
GRL1 method for the stiff variables combined with the FE method for
the non-stiff variables on the stiff subinterval of integration. We
refer to this new partitioned method as GRL1/FE$|$FE.



																%%%%%%%%%%%%%%
																% 4th column %
																%%%%%%%%%%%%%%

}
\end{pcolumn}
\begin{pcolumn}{0.24}
\pbox{0.9\textwidth}{66cm}{linewidth=2mm,framearc=0.1,linecolor=black,fillstyle=gradient,gradangle=0,gradbegin=white,gradend=white,gradmidpoint=1.0,framesep=1em}{

%%% Numerical Experiments
\begin{center}\pbox{0.8\textwidth}{}{linewidth=1mm,framearc=0.1,linecolor=black,fillstyle=gradient,gradangle=0,gradbegin=lightUofSgreen,gradend=lightUofSgreen2,gradmidpoint=1.0,framesep=1em}{\begin{center}\sc \textcolor{white}{\Large{Numerical Experiments}}\end{center}}\end{center}
\vspace{0.4cm}

To determine the efficiency of different numerical methods, we 
computed reference solutions with seven to ten matching digits
for the 37 cell models. Then we determined the maximum constant 
step sizes that satisfied a specific error tolerance for each of the 
models with respect to the reference solutions. We found that the FE
method wins on 9 models, the RL method wins on 24 models, and the
GRL1 method wins on 4 models:

\vspace{0.5 cm}
\begin{tabular}{|l| c | c | c |}
  \hline
   \bf{Basic Method}  & \bf{FE}  & \bf{RL} & \bf{GRL1} \\
\hline
 \bf{Winner on number of cell models} &  9 &  24  & 4 \\ 
\hline
\end{tabular}
\vspace{0.5 cm}
\\This confirms that
the popularity of the RL method in practice is well justified. The
secrets to its success lie mainly in its partitioning of the ODE
system into gating and non-gating variables and solving the equations
for the gating variables with an exponential integrator. The RL method
has the best combination of stability and computational expense per
step for moderately stiff models. Because the majority of the 37 cell
models are moderately stiff, the RL method is the best single method
for most models.

The performance of the partitioned method GRL1/FE$|$FE was determined
for five of the stiffest cell models. The execution time in seconds
of the basic methods (FE, RL, GRL1) and the partitioned method
GRL1/FE$|$FE is given in the following table:

\vspace{0.5 cm}
\begin{tabular}{|l| c | c | c | c |}
  \hline
   \bf{Model}  & \bf{FE}  & \bf{RL} & \bf{GRL1}  & \bf{GRL1/FE$|$FE} \\
\hline
 \bf{Bondarenko} &  2.23E+0 &  2.28E+0  & 8.41E--1   & \bf{9.59E--2}\\ 
\hline
 \bf{Jafri}  &  3.65E+0  & 3.59E+0 & 1.71E+1  & \bf{6.84E--1}  \\     
\hline
 \bf{Pandit}  &  5.55E+1 & 5.68E+1 &  9.67E--1  & \bf{2.10E--1}  \\ 
\hline
 \bf{Ten Tusscher Epi} &   1.29E+0 &  3.10E--2 & 1.67E--1   & \bf{2.55E--2}  \\  
\hline
 \bf{Winslow31} &  1.41E+1  &  1.49E+1 &  2.15E+2   & \bf{4.85E+0}  \\ 
\hline
\end{tabular}
\vspace{0.5 cm}
\\We see that GRL1/FE$|$FE is the most efficient method for all five of 
the stiff models considered. For the two stiffest models, namely that 
of Pandit et al. (2003) and Winslow31, GRL1/FE$|$FE is almost $5$ 
and $3$ times faster, respectively, than its next closest competitor.
GRL1/FE$|$FE is about $270$ and $3$ times faster than RL in these cases.
For the slightly less stiff models of Bondarenko et al. (2004) and
Jafri et al. (1998), GRL1/FE$|$FE is about $9$ and $5$
times faster than its next closest competitor, respectively. Finally 
for the Endocardial variant of the model of Ten Tusscher et al. (2006)
GRL1/FE$|$FE is $12$\% faster than its next closest competitor, RL. 

%These improvements can generally be understood as follows. In the
%stiff regions, the GRL1/FE$|$FE method can generally take larger step
%sizes than the RL method applied to the entire region because the
%partitioning of the ODEs better captures the stiffness for treatment
%by the exponential integrator than partitioning along the lines of
%gating vs.~non-gating variables.  Moreover because the use of GRL1 is
%limited to the relatively small number of stiff ODEs, it is also
%computationally cheaper per step than RL. In the non-stiff regions,
%GRL1/FE$|$FE reduces to FE, which is the cheapest method per step out
%of those considered. The GRL1/FE$|$FE method can also generally take
%larger steps on these regions than the corresponding steps for FE
%applied to the entire region because it is not impacted by
%restrictions from the stiff regions.



%%% Conclusion
\begin{center}\pbox{0.8\textwidth}{}{linewidth=1mm,framearc=0.1,linecolor=black,fillstyle=gradient,gradangle=0,gradbegin=lightUofSgreen,gradend=lightUofSgreen2,gradmidpoint=1.0,framesep=1em}{\begin{center}\sc \textcolor{white}{\Large{Conclusion}}\end{center}}\end{center}
\vspace{0.4cm}

Because of its overall efficiency and relative ease of implementation,
the Rush--Larsen method is a popular and effective method for solving
the ODEs that describe the evolution of dynamic myocardial cell
models. The Rush--Larsen method partitions the ODE system into gating
and non-gating variables and solves the equations for the gating
variables with an exponential integrator and the equations for the
non-gating variables with the forward Euler method. However, this
approach cannot be expected to work well on cell models for which the
stiffness is not captured by the gating variables. In this study we
demonstrate that in fact the stiffness in the stiffest cell models is
caused by non-gating variables, thus leading to underperformance of
the Rush--Larsen method. We demonstrate that a generalized
Rush--Larsen method of first order performs well on the stiffest cell
models. Using an eigenvalue analysis, we are able to partition the
ODEs and the interval of integration into stiff and non-stiff subsets
and hence propose a partitioned method based on the generalized
Rush--Larsen and forward Euler methods that outperforms all other
basic methods considered on the stiffest cell models.

{\footnotesize
\bibliographystyle{plain}
%\bibliography{explicitMethods}
}

}
\end{pcolumn}
\end{center}

\vspace*{2cm}

%%% Second Row %%%%%%%%%%%%%%%%%%%%%%%%%%%%%%%%%%%%%%%%%%%%%%%%%%%%%%%%%%%%%%%
%%% END Second Row %%%%%%%%%%%%%%%%%%%%%%%%%%%%%%%%%%%%%%%%%%%%%%%%%%%%%%%%%%%

\end{poster}

\end{document}

