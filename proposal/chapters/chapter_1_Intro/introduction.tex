
\chapter{Introduction}
\label{introduction}

Mathematical models are a gateway into understanding theoretical or experimental problems. By using such models, scientists are able to make new discoveries or design products for experimental testing. However, each model takes in a set of unknown parameters to produce a result, therefore to obtain the optimal result one needs the unknown parameters to achieve to it. Sometimes the optimal result(s) are unknown as well, optimization algorithms are applied to the model to obtain
optimal value(s) with the corresponding parameters.  
Two different types of methods for optimization, local and global. Local optimization searches in a given neighbourhood from a provided
candidate solution. In this region it converges to a local minima value based on its algorithmic properties. However, local optimization cannot guarantee a global minimum because its search space is determined by the neighbourhood  around the candidate solution. Likewise, a candidate solution cannot always be given, thus leading to the need of using a global optimizer.  Examples of the use of global optimization include chemical equilibrium, nuclear reactors, curve fitting, vehicle design and cost \cite{Pinter2002} and many more applications. 
Global optimization searches for an optimal solution in a user defined space until a user or
algorithmic stopping criteria is satisfied. The stopping criteria can be a length of time for the solver to run,
number of function evaluations or a termination condition.
Classification of global optimization algorithms are: deterministic, stochastic, heuristic, or machine learning \cite{Pinter2002}. These algorithms then have sub categories that are continuously growing with every group of problems needing to be solved.    
However, selecting one specific type of algorithm can not always guarantee a global minima because of the chance the solver will get stuck in a local minima. A deterministic method can typically surpass this challenge but it will require an undesirable amount of time to find the global minima. 
However, there are other methods of overcoming this challenge with out using a deterministic method. 
In this thesis, my contributions involve aiding global solvers by applying two software applications: computefarm
and a generalized database. 
Computefarm is a distributed system that utilizes unused computer resources on various client computers to run multiple function evaluations and can speed up the iteration process of the given solver. By doing this we speed up the process of finding the global minima in a black box problem with any open source global solver. 
The generalized database is a simplified database that allows the model or solver to store any user defined results from this the user can obtain the status of the solver. By knowing this the user can: i) determine if the solver is stuck in a local minima, ii) start other solvers based on current stored data or iii) use the data to further analyse the model. 
The applications studied in this thesis that used computefarm and the generalized database for their solution are quantum error correction circuit design and diatomic crystal prediction. 
Quantum error correction can not be controlled using a software based design like that of a transistor based computer, instead it is controlled directly from a circuit component. However, when designing circuits and experimentally testing them for desired reliability this becomes costly for both the manufacturer and the experimental testing. Therefore, several models have been designed to simulate a quantum error correction circuit to determine the effect of error correction on given n-qubit system. To ensure the best feasible
reliability of error correction, the circuit design model is optimized to a feasible solution. However, as the number of qubits in the system increase the complexity of finding a feasible solution becomes more difficult because the number circuit parameters increase. This problem is then optimized globally because a candidate solution cannot be determined due to the increased complexity there are multiple local minima in the search space for the solver to be stuck on. In this thesis we
solve the application for the three qubit and four qubit system by applying a global optimizer using the generalized database to solve various cases. 
Crystal structure prediction has been used in the past decade to predict the most stable structure at a given temperature and pressure environment. Because there is a large variety of lattice positions for a given compound, experimentally testing each possible lattice structure at all desired temperatures and pressures can be taxing. Ab initio structure codes are used to calculate properties of specific lattices to aid in the discovery of new structures or the understanding of current
ones. One specific property that these codes can determine is the total energy of the lattice structure, which represents the stability of the structure in a given environment. The lower the energy, the more stable the structure is and the greater potential for it being generated experimentally. One specific area where this is useful is in studying the crystal generation for various applications like electronic devices, protein research and material design. The challenge with this application is that their large
number of lattice structure arrangements and the code to calculate their properties take an upward to twenty to thirty minutes for complex compounds. This makes attempts globally optimize a crystal structure undesirably slow. Another challenge is that meta-stable structures can have more useful properties than most stable structures and therefore the optimization needs to list also the meta-stable structures or that are found. In this thesis we focus on the diatomic structure silicon
dioxide, known for having various quartz compositions that are used in mechanical devices like watches as well microprocessors because of their ability to generate and hold electrical charge. By optimizing this structure, an experimentalist can determine various properties that prove to be superior in various applications than other types of quartz. This specific application uses the generalized database to store various types of quartzes structures and used the computefarm to
decrease the total time required for optimizing the structure.
In this these is, Chapter \ref{background} is the background on global optimization, Chapter \ref{methods} documents the software developed to aid in the global optimization using computefarm and the generalized database, Chapter \ref{applications} is the description of the applications quantum error correction circuit design and crystal structure prediction of silicon dioxide, Chapter \ref{conclusion} concludes the results of the applications.
