\abstract{
    Mathematical models are a gateway into understanding theoretical and experimental occurrences. However, to understand these models certain parameters need to be established in order to obtain the optimal behaviour or value. Global optimization is a branch of optimization that takes a function and maximizes (or minimizes) the value within a given search space. Researchers over the past decades have developed numerous global optimization algorithms for solving problems by using various designs.
    However, global optimizations can be hindered when the search space yields multiple local minima, as this can cause many solvers to become stuck in the search for a global minimum value. Moreover, the complexity of the mathematical model and the lengths of calculations tend to increase the amount of time required for the solver to evaluate the model. To address these challenges, two pieces of software were developed: Computefarm and Checkpointing. Computefarm is a distributed system that parallelizes the iteration step of a solver by distributing function evaluations to farmed computers. Checkpointing is a robust database schema that prevents data from being lost as a result of optimization failure. It is also used to monitor and manipulate both solvers and function evaluations, so as to speed up the search for the global value. 
In this thesis, Computefarm and Checkpointing were studied in the context of two particular applications. The first is designing quantum error correction circuits. Quantum computers cannot rely on software to correct errors because of the quantum mechanical properties of subatomic particles. Instead, circuits are designed to correct for different types of errors. To ensure a high fidelity of the circuit, a simulation of a circuit's fidelity can be optimized using the Checkpointing software to obtain a circuit design for the four qubit and three qubit cases with a fidelity of $99.99\%$.

The second application is crystal structure prediction, which is done by using an ab initio code called VASP that obtains the total energy of given lattice structure. This determines the stability of the structure; the stable structures of diamond or graphite, for example, can be predicted with this technique. For more complex structures like that of silicon dioxide, however, the simulations are usually more complex and thus take longer to run. By applying Computefarm with a solver, multiple simulations can be run at the same time to produce the teragonal structure of silicon dioxide. 

Through their applications to quantum eror correction and crystal structure prediction, both Computefarm and Checkpoint are shown to be useful software for optimizing problems in shorter periods of time. 

