\abstract{
    Mathematical models are a gateway into understanding theoretical and experimental. However, sometimes these models need certain parameters to be established in order to obtain the optimal behaviour or value. Global optimization is a branch of optimization that takes a function and minimizes the value within a given search space.
    However, global optimizations can be come extremely challenging when the search space yields multiple local minima. Moreover, the complexity of the mathematical model and the consequent lengths of calculations tend to increase the amount of time required for the solver to evaluate the model.
    To address these challenges, two pieces of software were developed that aided the solver to optimize a black box problem. First software developed is Computefarm a distributed system that parallelizes the iteration step of a solver by distributing function evaluations to unused computers. The Second software is a general database schema that prevents data from being lost as a result of an optimization failure, it is also used to monitor and manipulate both solvers and function evaluations. 

    In this thesis, both Computefarm and the general database were studied in the context of two particular applications. The first is designing quantum error correction circuits. Quantum computers cannot rely on software to correct errors because of the quantum mechanical properties of the quantum bits. Instead, circuits are designed to correct for different types of errors. To ensure a high fidelity of the circuit, a simulation of a circuit's fidelity can be optimized using the general database
    software to obtain a circuit design for the  three qubit and four qubit cases with a fidelity of $99.99\%$. The second application is crystal structure prediction that minimizes the total energy of crystal lattice to obtain its stability. From doing this a stable structure for silicon dioxide is obtained; however, the simulations take longer to run. By applying Computefarm with a solver, multiple simulations can be run at the same time to produce the teragonal structure of silicon
    dioxide.}

