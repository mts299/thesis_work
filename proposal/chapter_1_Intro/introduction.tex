
\chapter{Introduction}
\label{introduction}

Mathematical models are a gateway into understanding both theoretical and experimental problems. By using such models, scientists are able to make new discoveries or design products for experimental testing. However, each model takes in a set of unknown parameters to produce a result, and therefore to obtain the optimal result one needs the unknown parameters. Sometimes the optimal result(s) are unknown as well. Optimization algorithms are applied to the model to obtain optimal value(s) within the corresponding parameters.  

There are two different types of methods for optimization: local and global. Local optimization searches in a given neighbourhood around a provided candidate solution. In this region it converges to a local minimum value based on its algorithmic properties. However, local optimization cannot guarantee a global minimum because its search space is determined by the neighbourhood around the candidate solution. Likewise, a candidate solution cannot always be given, creating the need for a global optimizer.  Examples of the application of global optimization include chemical equilibrium, nuclear reactors, curve fitting, vehicle design and cost \cite{Pinter2002}, and many others.
Global optimization searches for an optimal solution in a user defined space until a user or an algorithm-stopping criterion is satisfied. The stopping criterion can be a length of time for the solver to run, a number of function evaluations or a termination condition. Global optimization algorithms can be classified as deterministic, stochastic, heuristic, or machine learning \cite{Pinter2002}. These algorithms have further subcategories that continue to grow with every group of problems needing to be solved. 

Selecting one specific type of algorithm cannot always guarantee a global minimum because of the chance that the solver will get stuck at a local minimum. A deterministic method can typically overcome this challenge, but it will require a substantial amount of time to find the global minimum. Fortunately, there are other (non-deterministic) methods that can resolve the issue. 

This thesis will contribute to the success of global solvers through the application of two pieces of software: computefarm and a generalized database. Computefarm is a distributed system that utilizes unused computer resources on various client computers to run multiple function evaluations. As a result, it can speed up the iteration process of the given solver, and thereby speed up the search for the global minimum in a black-box problem with any open-source global solver. 

The generalized database is a simplified database that allows the model or solver to store any user-defined results, from which the user can obtain the status of the solver. With this information the user can i) determine if the solver is stuck at a local minimum, ii) initiate other solvers based on currently stored data, or iii) use the data to further analyse the model. 

The applications studied in this thesis that used computefarm and the generalized database for their solution are quantum error correction circuit design and diatomic crystal prediction. Quantum error correction cannot be controlled using a software-based design like that of a transistor-based computer. Instead, it is controlled directly from a circuit component. This makes the process of manufacturing circuits and testing them for desired reliability quite costly. In light of this, several models have been designed to simulate a quantum error correction circuit to determine the effect of error correction on a given n-qubit system. To ensure the most feasible reliability of error correction, the circuit design model is optimized to a feasible solution. However, as the number of qubits in the system increases, so does the number of circuit parameters. This makes the process of finding a feasible solution more complex, as there are multiple local minima in the search space for the solver to become stuck on. Since a candidate solution may not be determinable in these circumstances, this problem needs to be optimized globally. In this thesis we apply a global optimizer using the generalized database to find solutions for the three-qubit and four-qubit systems. 

Crystal structure prediction has been used in the past decade to approximate the most stable structure of a compound at a particular temperature and pressure environment. Because there is a large variety of lattice positions for any given compound, testing every possible lattice structure at all desired temperatures and pressures can be taxing. Ab initio structure codes are used to calculate properties of specific lattices to aid in the discovery of new structures or the understanding of current
ones. One specific property that these codes can determine is the total energy of the lattice structure, which represents the stability of the structure in a certain environment. The lower the energy, the more stable the structure and the greater potential for its being generated experimentally. One specific area where this is useful is the study of crystal generation for various applications including electronic devices, protein research and material design. 

The challenge with this application is that the large number of lattice structure arrangements and the code used to calculate their properties take upward to twenty or thirty minutes for complex compounds. This makes attempts to globally optimize a crystal structure undesirably slow. Another challenge is that meta-stable structures can have more useful properties than most stable structures and therefore the optimization needs to list the meta-stable structures in addition to the stable structures that are found. 
In this thesis we focus on the diatomic structure of silicon dioxide, known for having various quartz compositions that are used in mechanical devices like watches and microprocessors because of their ability to generate and hold electrical charge. By optimizing this structure, an experimentalist can determine various properties that prove to be superior in certain applications to other types of quartz. This specific application uses the generalized database to store various types of quartz structures, and uses the computefarm software to decrease the total time required for optimizing the structure.

In this thesis, Chapter \ref{background} gives the background on global optimization, Chapter \ref{methods} documents the software developed to aid in global optimization using computefarm and the generalized database, Chapter \ref{applications} describes the two applications, quantum error correction circuit design and crystal structure prediction of silicon dioxide, and Chapter \ref{conclusion} concludes with the results of the applications.
