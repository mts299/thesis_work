\abstract{
    Mathematical models are a gateway into understanding theoretical and experimental. However, sometimes these models need certain parameters to be established in order to obtain the optimal behaviour or value. Optimization 
 is a method to obtain certain parameters for optimal behaviour that may be a minimum (or maximum) result. Global optimization is a branch of optimization that takes a function and determines the global minimum for a given domain.
    However, global optimizations can be come extremely challenging when the search space yields multiple local minima. Moreover, the complexity of the mathematical model and the consequent lengths of calculations tend to increase the amount of time required for the solver to evaluate the model.
    To address these challenges, two pieces of software were developed that aided the solver to optimize a black box problem. First software developed is Computefarm a distributed system that parallelizes the iteration step of a solver by distributing function evaluations to unused computers. The second software is a Optimization Database that prevents data from being lost as a result of an optimization failure, it is also used to monitor the function and to store extra information on
    function and results. 

    In this thesis, both Computefarm and the Optimization Database were in the context of two particular applications. The first is designing quantum error correction circuits. Quantum computers cannot rely on software to correct errors because of the quantum mechanical properties that allow non-deterministic behaviour in the quantum bit. This means the quantum bits can change states between $(0,1)$ at any point in time. There are various ways to stabilize the quantum bits however,
    errors in the system of quantum bits and system to measure the states can occur. Therefore, error correction components are designed to correct for these different types of errors, to ensure a high fidelity of the overall circuit. A simulation of a quantum error correction circuit is used to determine the properties of components needed to stabilize to obtain minimal error. This simulation is optimized with the use of the Optimization database and Computefarm to
    obtain the properties need to achieve a low error rate. 
    The second application is crystal structure prediction that minimizes the total energy of crystal lattice to obtain its stability. From doing this stable structures of Carbon and silicon dioxide is obtained by using Computefarm and the Optimization Database to obtain and store various stable structures and post processing data for classification of the structure. The use of the database then is used to obtain the top $N$ stable structures of Carbon and Silicon Dioxide for further research to be
  done on.}

