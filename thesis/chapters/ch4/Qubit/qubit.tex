Quantum computers are promising technology that is currently being used today by D-Waves, Nasa, google and IBM. Multiple institutes also focus on researching quantum computers and their exponential ability to represent information. This is obtained from the representation of a quantum bit also known as a \textit{qubit}, a qubit is subatomic particle that is measured based on a known property like spin or an energy state. These properties represent states zero or one, same as a
classical bit. However, qubits also have \textit{superposition} or \textit{entangled} states that the known states become linear combinations. For example in a classical computer two bits can represent one of the four possible values
\begin{align}
    00,\\
    10,\\
    01,\\
    11.
\end{align}
To make up these combinations it takes two pieces of information, the first bit value and the second, thus a classical computer takes $N$ bits of information. In a quantum computer two qubits can represent one of the same possible classical computer values. However, qubits are non-deterministic meaning they can be in any combination of states at any point in time known as \textit{superposition}. Therefore to encode the qubits into a specific states, probabilities are
given to the qubit system 
\begin{align}
    \alpha |00 \rangle,\\
    \beta  |10 \rangle,\\
    \gamma |01 \rangle,\\
    \delta |11 \rangle.\\
\end{align}
Thus four pieces of information are given to system giving quantum computers the exponential advantage of $2^N$ ($N$ qubits) bits of information. Because of this advantage, a combination of calculation steps are done simultaneously or reduced to less steps with the use of more information. A strong example of this is prime factorization, where it is NP-hard problem for classical computers. In quantum computers the number fifteen has been solved using the Shor's algorithm
on a three qubit system in seconds by Lucero \cite{Lucero2013}. With Dattani and Bryans \cite{Dattani2014} factorizing the number 56153 using a four qubit system. However, as it is mentioned in both papers Lucero \cite{Lucero2013} and Dattani \cite{Dattani2014} the error of the qubit systems prevents further work into factorization of higher numbers. With high error rates the success rate becomes lower. This becomes a difficult challenge for quantum cryptography that
utilizes the fact quantum computers can factor high prime numbers in an efficient amount of time. However, cryptography needs fault tolerance to ensure the security of encryption and the ability to decrypt.
To insure fault tolerance error correction components have been design to correct for any error in the qubit system. The two main errors in a qubit system is decoherence and measurement error. \textit{Decoherence} is the loss of energy in the qubit system caused by interaction background particles. When two particles interact unintentionally this causes a loss of energy in the system, this energy is the information passed into the system, when the energy is lost the information is
too. The other error that can occur is measurement error, this is caused by noise in the system. To correct for these errors pulses of energy are used to stabilize the qubit system because cloning qubit states is not possible due to their non-deterministic behaviour. However, stabilizing the system is constricted to specific amount of time because of decoherence. When enough outside interactions happen the qubit system can no longer be stabilized as their is too much loss of information.
Another challenge is decoherence is not easy to model as it is not easy to predict what background particles will interact with the qubits and how the interaction will affect the system. Noise error is easier to model as it is the noise in qubit system and in logic gate utilizing the system. Therefore by optimizing the error correction circuit design to have a qubit system simulated to have noise error be fault tolerant, then experimentally that circuit can be further optimized for decoherence.

In this section the Controlled Controlled Controlled Not (CCCNOT) gate is optimized to obtain a feasible error correction percentage known as \textit{intrinsic fidelity}. A simulation of the CCCNOT gate is optimized to achieve a feasible intrinsic fidelity of $99.99\%$ to guarantee the highest on average experimental gate fidelity of $99.9\%$ \cite{Barends2014}. The CCCNOT simulates a four superconducting charge qubits known as \textit{transmons}. Transmons are used because of the reduced sensitivity to charge
noise to aid in reducing error. The states zero and one are represented as energy states of the transmon, $j$. Each transmon location in the system is represented as $k$ that receive pulses from the error correction circuit over a given amount of time, $t$. The shift in frequencies sent to each transmon is represented as $\Delta_k(t)$ (bounded between $-2.5$ and $2.5$ MHz) and anharmonicity of the shift in frequency of each transmon is represented as $\eta_{jk}$ that is measured to be $200$ MHz for the circuit. The
energy of each $k^{th}$ transmon at each energy level $j$ is
\begin{equation}
    \label{eq:energy}
    E_{kj} = h(j\Delta_k(t)-\eta_{jk}),
\end{equation}
where $h$ is planks constant. As mentioned earlier qubits can entangle with one another, this interaction is represented as a nearest-neighbour coupling strength, $g_k$, between each $kth$ and $(k+1)th$ transmon. The coupling strength is set as $30$ MHz in simulation.  

The energy transition between transmons states is then represented as the $n$ transmon system (in the case of a four qubit system $n=4$)
\begin{equation}
  \label{eq:hamiltonian}
  \frac{\hat{H}\big(\delta_k(t)\big)}{h} = \sum^n_{k=1} \sum^n_{j=0} E_{kj} |j \big\rangle_k \big\langle j|_k + \sum^{n-1}_{k=1} \frac{g_k}{2}(X_kX_{k+1}+Y_{k}Y_{k+1}),  
\end{equation}

where $X$ and $Y$ are the coupling operators \cite{Ghosh2013}
\begin{align}
    \label{eq: coupling operators}
        X_k = \sum^n_{j=1} \sqrt{j}\big| j-1 \big\rangle_k \big\langle j |_k + hc, \\
        Y_k = -\sum^n_{j=1} \sqrt{-j}\big| j-1 \big\rangle_k  \big\langle j |_k + hc,
\end{align}
and $hc$ is the Hermitian conjugate. The couple operators \eqref{eq: coupling operators} represent the generalize Pauli spin matrices for each $kth$ transmon.

By knowing the Hamiltonian \eqref{eq:hamiltonian} of the transmons system the evolution operator of the system over a time $t$ is represented as
\begin{equation}
  \label{eq:evolution operator}
  U\big( \Delta_k(\Theta) \big) = T e^{\Big\{ -i \int_{0}^{\Theta} \hat{H}\big( \Delta_k(t) \big) dt \Big\} }, 
\end{equation}
Where $\Theta$ is time duration of the error correction and $T$ is the time ordered evolution operator \cite{}. 

Because transmons states have to fall between zero or one (bias states) when observed, high energy levels are not considered when correcting the error. To do this a projection is taken on the evolution operator \eqref{eq:evolution operator} to obtain the computation subspace
\begin{equation}
    \label{eq:projected}
    U_{\mathscr{P}} \big(\Delta_k (\Theta) \big) =  \mathscr{P} U \big(\Delta_k(\Theta) \big) \mathscr{P}.
\end{equation}

The computational subspace \eqref{eq:projected} is the performance metric intrinsic fidelity. To simplify this down to a single percentage value, the percentage of the intrinsic fidelity is represented as 
\begin{equation}
    \mathscr{F}\big(\Delta_k(\Theta)\big)=\frac{1}{N}\Bigg| \mathrm{Tr}\bigg( CCCNot^{\dagger} U_\mathscr{P}\big(\Delta_k(\Theta) \big) \bigg) \Bigg|,
\end{equation}
where CCCNot is the ideal gate \cite{}. 

This model represent the objective function to optimize the frequency shifts, $\Delta_k(t)$, for a $n$-transmons system to obtain feasible intrinsic fidelity of $99.99\%$ for a minimal duration time $\Theta$.

The optimization problem is presented as follows
\begin{equation}
    \label{eq:feasibility}
    0.9999 \leq f(x),
\end{equation}
where a feasible solution is just needed. 

To solve this problem for the four-qubit ($n=4$) and three-qubit ($n=3$) case we used \textit{MATLABs} global search algorithm from the global optimization toolbox with a non-linear constraint to represent the feasibility condition \eqref{eq:feasibility}. To minimize the duration time of the simulation a brute force method is used by simply solving each case with a duration time and lowering it to find the minimal value. The Optimization Database is used in the optimization process to
monitor multiple duration time cases and the progress of global optimization. By using this method to solve the problem the following results for the four-qubit case are shown in Table~\ref{tbl:four qubit} and the three qubit case result shown in Table~\ref{tbl:three qubit}. 



The feasible pulse for minimal duration time of $65$ nanoseconds for the four qubit case is shown in Figure~\ref{fig:four qubit}.


The feasible pulse for minimal duration time of $23$ nanoseconds for the three qubit case is shown in Figure~\ref{fig:three qubit}.


By obtaining these results, error correction circuits using the CCCNot gate are developed for further experimental optimization to account for decoherence error to be used in quantum computers.
