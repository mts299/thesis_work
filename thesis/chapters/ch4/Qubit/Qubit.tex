\section{Quantum error correction circuit design}
\label{Qubit}
Quantum computers are promising technology that is currently being used today by D-Waves, Nasa, google and IBM. Multiple institutes also focus on researching quantum computers and their promise to solve non-polynomial hard (NP hard) problems. One problem of interest is prime factorization because its exponential time complexity in a transistor computer versus the polynomial time using the Shor's algorithm in quantum computer. To execute this algorithm, it is assumed extra bit states are
available to be used in the algorithm for transistor computer this is not possible as it is constrained to zero or one for bit states. However, in a quantum computer the number of states is dependent sub atomic particle property used to represent a bit also known as a \textit{qubit}. A qubit states can be based on a measurement of spin or energy states. These properties represent states zero or one, same as a
transitor bit. However, qubits also have \textit{superposition} or \textit{entangled} states that are represented as linear combinations. For example in a transitor computer two bits can represent one of the four possible values
\begin{align*}
    00,\\
    10,\\
    01,\\
    11.
\end{align*}
To make up these combinations it takes two pieces of information, the first bit value and the second, thus a transitor computer takes $N$ bits of information. In a quantum computer two qubits can represent zero and one, and any combination of states at any point in time known as \textit{superposition}. Therefore to encode the qubits into a specific state, probabilities are
given to the qubit system 
\begin{align*}
    \alpha |00 \rangle,\\
    \beta  |10 \rangle,\\
    \gamma |01 \rangle,\\
    \Delta |11 \rangle,\\
\end{align*}
to produce a linear combination of $\alpha |00 \rangle + \beta  |10 \rangle + \gamma |01 \rangle + \Delta |11 \rangle.$
Thus four pieces of information are given to the system giving quantum computers the exponential advantage of $2^N$ ($N$ qubits) bits of information. Because of this advantage, a combination of calculation steps are done simultaneously or reduced to less steps with the use of more information. Giving the advantage to solving NP hard problems in less amount of time than transistor computers. Relating back to the prime factorization problem, factoring the number fifteen has been using the Shor's algorithm
on a three qubit system is done in seconds by Lucero \cite{Lucero2013}, with Dattani and Bryans \cite{Dattani2014} factorizing the number i$56153$ using a four qubit system . However, as it is mentioned in both papers Lucero \cite{Lucero2013} and Dattani \cite{Dattani2014} the error of the qubit systems prevents further work into factorization of higher numbers. With high error rates the success rate becomes lower and the qubit system cannot be cloned to later on check and correct for
error like a transistor computer. Because qubits utilize superposition their states are not guaranteed to be the same at any given point, making it difficult to guarantee $100\%$ error correction rate. However, methods have been developed to minimize and correct errors in a quantum computer to ensure a high fault tolerance. One method that is looked at in this thesis is the controlled-Z gate that uses control-phase gates to control one of the errors in the qubit system. 

Three error that occur control-phase gates that are used simulated in model are \textit{decoherence}, \textit{state leakage} and \textit{control time} \cite{Barends2014}. The first error, decoherence, occurs when outside particles interact with the qubit system particles. When these particles interact it causes a change in energy of the qubit that further interferes with the superposition of the qubit system. This interference over time makes the qubit system loss information of its original
state and causes error in the result. To minimize this error, certain materials can be used to protect the qubit system from incoming particles to reduce interactions
The other error that can occur is state leakage, this error occurs when higher energy level states are measured that are not desired. Because the qubit system is setup to take in probabilities for subspace that only contains zeros and ones known as a \textit{computational subspace}, then measured states of higher energies than one are not associated to original input. To avoid this from happening, large gate times for measuring energies are used because qubits tend to transition to lower
energy states, zero or one, over time.
The final error mentioned is the control error, this error occurs in the gate itself, where reflections or stray inductance in wiring effects the qubit system. Because qubits are controlled by frequencies that change the properties of the particle, inductance of wire can disturb this control and cause error in the system. To correct for this error, low frequency pulses are sent in to stabilize the qubit system, this pulse is known as a \textit{controlled} pulse. However, this becomes a
circuit design problem that can be simulated with a n-qubit system to be optimized to produce a controlled pulse that correct for the control error. The fault tolerance that only considers control error is known as \textit{intrinsic fidelity}. Because control error is only considered in the optimization of the circuit design, an intrinsic fidelity of $99.99\%$ is needed. Anything higher and it becomes a potential waste of computational time because experimentally the highest intrinsic
fidelity is $99.9\%$ due to limitations on handling decoherence. 

In this section the controlled-Z (CZ) gate is optimized to obtain an intrinsic fidelity of $99.99\%$ for a three and four qubit system. The qubit systems use superconducting charge qubits known as \textit{transmons}. Transmons are used because of the reduced sensitivity to charge
noise to aid in reducing error. The states zero and one are represented as energy states of the transmon, $j$. Each transmon location in the system is represented as $k$ that receive a pulse from the error correction circuit over a given amount of time, $t$. The shift in frequencies sent to each transmon is represented as $\Delta_k(t)$, where each pulses frequency is varied between $-2.5$ and $2.5$ MHz. The anharmonicity of the shift in frequency of each transmon is represented as $\eta_{jk}$ that is set to $200\ MHz$ for the circuit. The
energy of each $k^{th}$ transmon at each energy level $j$ is
\begin{equation}
    \label{eq:energy}
    E_{kj} = h(j\Delta_k(t)-\eta_{jk}),
\end{equation}
where $h$ is planks constant. As mentioned earlier qubits can entangle with one another, this interaction is represented as a nearest-neighbour coupling strength, $g_k$, between each $kth$ and $(k+1)th$ transmon. The coupling strength is set as $30$ MHz in the simulation.  

The energy transition between each transmon is represented as a $j^n$ block-diagonal Hamiltonian matrix
\begin{equation}
  \label{eq:hamiltonian}
  \frac{\hat{H}\big(\Delta_k(t)\big)}{h} = \sum^n_{k=1} \begin{pmatrix} 0&0& 0&0\\0&\varepsilon_k(t)&0&0\\0&0&2\varepsilon_k(t)-\eta&0\\0&0&0&3\varepsilon_k(t)-\eta'\end{pmatrix}_{k}  + \sum^{n-1}_{k=1} \frac{g_k}{2}(X_kX_{k+1}+Y_{k}Y_{k+1}),  
\end{equation}
where each block corresponds to fixed number of excitations, that acts on Hilbert space $\mathscr{H}_4^{\otimes}n$. The $X$ and $Y$ are the coupling operators \cite{Ghosh2013}
\begin{align}
    \label{eq: coupling operators}
       X_ki &=& \begin{pmatrix} 0&1& 0&0\\1&0&\sqrt{2}&0\\0&\sqrt{2}&0&\sqrt{3}\\0&0&\sqrt{3}&0\end{pmatrix}_{k}, \quad
  \frac{Y_k}{\text i} &=&\begin{pmatrix} 0&-1& 0&0\\1&0&-\sqrt{2}&0\\0&\sqrt{2}&0&-\sqrt{3}\\0&0&\sqrt{3}&0\end{pmatrix}_{k}, 
\end{align}
are the generalized Pauli operators. The Hamiltonian is then truncated up to the $n^{th}$ excitation subspace
\begin{align}
  \label{eq: ham}
  \hat{H}_p(\Delta(t)) = \mathscr{O}_n \hat{H}(\Delta(t)) \mathscr{O}_n^{\dagger},
  \end{align}
  because it is the highest excitation that can be achieved in the computational space.
  The Hamiltonian system \eqref{eq: ham} is then evolved over a time $t$
\begin{equation}
  \label{eq:unitary operator}
  U\big( \Delta_k(\Theta) \big) = T e^{\Big\{ -i \int_{0}^{\Theta} \hat{H}\big( \Delta_k(t) \big) dt \Big\} }, 
\end{equation}
Where $\Theta$ is the duration time of the error correction and $T$ is the time-ordering evolution operator \cite{}. 

Because it is assumed in this system that the qubit system is measured in the computation subspace, the Unitary operator \eqref{eq:unitary operator} is projected to computational subspace to give a $2^n$ matrix
\begin{equation}
    \label{eq:projected}
    U_{\mathscr{P}} \big(\Delta_k (\Theta) \big) =  \mathscr{P} U \big(\Delta_k(\Theta) \big) \mathscr{P}^{\dagger}.
\end{equation}

This matrix \eqref{eq:projected} is then compared to the ideal target CZ gate 
\begin{align}
  \label{eq:ideal}
  U_{\text{target}} = U_l U_{\text{cz}} U_r,
\end{align}
where the ideal CZ gate, $U_{\text{cz}}$ is
\begin{align}
  U_{\text{cz}} = \begin{pmatrix}
                    1 & 0 & 0 & 0 \\
                    0 & 1 & 0 & 0 \\
                    0 & 0 & 1 & 0 \\
                    0 & 0 & 0 & -1 \\
                  \end{pmatrix}
\end{align}
and the Unitary phase shift on the $z$-axis, $U_{l,r}$, is
\begin{align}
  U_{l,r} =& R_z(\phi_1) \otimes R_z(\phi_2) \otimes R_z(\phi_3) \otimes \cdots \otimes R_z(\phi_k),\\
  R_z(\phi) =& \begin{pmatrix} 1 & 0 \\ 0 & e^{-i\phi_k} \end{pmatrix}, 
\end{align}
to obtain the metric of performance in error correction.

The intrinsic fidelity is then obtained by
\begin{equation}
  \label{eq:fidelity}
    \mathscr{F}\big(\Delta_k(\Theta)\big)=\frac{1}{N}\Bigg| \mathrm{Tr}\bigg( CZ^{\dagger} U_\mathscr{P}\big(\Delta_k(\Theta) \big) \bigg) \Bigg|.
\end{equation}

This model \eqref{eq:fidelity} represents the objective function to optimize the frequencies, $\Delta_k(t)$, for a $n$-transmons system to obtain feasible intrinsic fidelity of $99.99\%$ for a minimal duration time $\Theta$.

The optimization problem is presented as follows
\begin{equation}
    \label{eq:feasibility}
    0.9999 \leq f(x),
\end{equation}
where a single feasible solution is desired. 

In this thesis the four-qubit and three-qubit case are optimized using \textit{MATLABs} global search algorithm from the global optimization toolbox with a non-linear constraint to represent the feasibility condition \eqref{eq:feasibility}. To minimize the duration time of the simulation a brute force method is used by simply solving each case with a smaller duration time each time the desired fidelity is achieved. The Optimization Database is then used in the optimization process to
monitor multiple duration time cases and the progress of the global optimization. By using this method to solve the problem the following results for the four-qubit case are shown in Table~\ref{tbl:four qubit} and the three qubit case result shown in Table~\ref{tbl:three qubit}. 

\begin{table}[!h]
  \label{tbl:four qubit}
  \centering
\begin{python}
from data_representation import LatexData

lb = LatexData('t_4qubit')
lb.generate_table(3,'T',['Duration time $\Theta$','Fidelity'])

\end{python}
\caption{Duration time results for intrinsic fidelity for the four qubit case.}
\end{table}

\begin{table}[!h]
  \label{tbl:three qubit}
  \centering
\begin{python}
from data_representation import LatexData

lb = LatexData('t_3qubit')
lb.generate_table(3,'T',['Duration time $\Theta$','Fidelity'])

\end{python}
\caption{Duration time results for intrinsic fidelity for the four qubit case.}
\end{table}



The feasible pulse for minimal duration time of $70$ nanoseconds for the four qubit case is shown in Figure~\ref{fig:four qubit}.



The feasible pulse for minimal duration time of $24$ nanoseconds for the three qubit case is shown in Figure~\ref{fig:three qubit}.


By obtaining these results, error correction circuits using the CZ gate are developed for further experimental optimization to account for decoherence error to be used in quantum computers.
