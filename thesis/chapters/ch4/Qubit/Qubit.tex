\section{Quantum error correction circuit design}
\label{Qubit}
Quantum computers are promising technology that is currently being used today by D-Waves, Nasa, google and IBM. Multiple institutes also focus on researching quantum computers and their promise to solve non-polynomial hard (NP hard) problems. One problem of interest is prime factorization because its exponential time complexity in a transistor computer versus the polynomial time using the Shor's algorithm in quantum computer. To execute this algorithm, it is assumed extra bit states are
available to be used in the algorithm for transistor computer this is not possible as it is constrained to zero or one for bit states. However, in a quantum computer the number of states is dependent sub atomic particle property used to represent a bit also known as a \textit{qubit}. A qubit states can be based on a measurement of spin or energy states. These properties represent states zero or one, same as a
transitor bit. However, qubits also have \textit{superposition} or \textit{entangled} states that are represented as linear combinations. For example in a transitor computer two bits can represent one of the four possible values
\begin{align*}
    00,\\
    10,\\
    01,\\
    11.
\end{align*}
To make up these combinations it takes two pieces of information, the first bit value and the second, thus a transitor computer takes $N$ bits of information. In a quantum computer two qubits can represent zero and one, and any combination of states at any point in time known as \textit{superposition}. Therefore to encode the qubits into a specific state, probabilities are
given to the qubit system 
\begin{align*}
    \alpha |00 \rangle,\\
    \beta  |10 \rangle,\\
    \gamma |01 \rangle,\\
    \delta |11 \rangle,\\
\end{align*}
to produce a linear combination of $\alpha |00 \rangle + \beta  |10 \rangle + \gamma |01 \rangle + \delta |11 \rangle.$
Thus four pieces of information are given to the system giving quantum computers the exponential advantage of $2^N$ ($N$ qubits) bits of information. Because of this advantage, a combination of calculation steps are done simultaneously or reduced to less steps with the use of more information. Giving the advantage to solving NP hard problems in less amount of time than transistor computers. Relating back to the prime factorization problem, factoring the number fifteen has been using the Shor's algorithm
on a three qubit system is done in seconds by Lucero \cite{Lucero2013}, with Dattani and Bryans \cite{Dattani2014} factorizing the number i$56153$ using a four qubit system . However, as it is mentioned in both papers Lucero \cite{Lucero2013} and Dattani \cite{Dattani2014} the error of the qubit systems prevents further work into factorization of higher numbers. With high error rates the success rate becomes lower and the qubit system cannot be cloned to later on check and correct for
error like a transistor computer. Because qubits utilize superposition their states are not guaranteed to be the same at any given point, making it difficult to guarantee $100\%$ error correction rate. However, methods have been developed to minimize and correct errors in a quantum computer to ensure a high fault tolerance. One method that is looked at in this thesis is the controlled-Z gate that uses control-phase gates to control one of the errors in the qubit system. 

Three error that occur control-phase gates that are used simulated in model are \textit{decoherence}, \textit{state leakage} and \textit{control time} \cite{Barends2014}. The first error, decoherence, occurs when outside particles interact with the qubit system particles. When these particles interact it causes a change in energy of the qubit that further interferes with the superposition of the qubit system. This interference over time makes the qubit system loss information of its original
state and causes error in the result. To minimize this error, certain materials can be used to protect the qubit system from incoming particles to reduce interactions
The other error that can occur is state leakage, this error occurs when higher energy level states are measured that are not desired. Because the qubit system is setup to take in probabilities for subspace that only contains zeros and ones known as a \textit{computational subspace}, then measured states of higher energies than one are not associated to original input. To avoid this from happening, large gate times for measuring energies are used because qubits tend to transition to lower
energy states, zero or one, over time.
The final error mentioned is the control error, this error occurs in the gate itself, where reflections or stray inductance in wiring effects the qubit system. Because qubits are controlled by frequencies that change the properties of the particle, inductance of wire can disturb this control and cause error in the system. To correct for this error, low frequency pulses are sent in to stabilize the qubit system, this pulse is known as a \textit{controlled} pulse. However, this becomes a
circuit design problem that can be simulated with a n-qubit system to be optimized to produce a controlled pulse that correct for the control error. The fault tolerance that only considers control error is known as \textit{intrinsic fidelity}. Because control error is only considered in the optimization of the circuit design, an intrinsic fidelity of $99.99\%$ is needed. Anything higher and it becomes a potential waste of computational time because experimentally the highest intrinsic
fidelity is $99.9\%$ due to limitations on handling decoherence. 

In this section the controlled-Z (CZ) gate is optimized to obtain an intrinsic fidelity of $99.99\%$ for a three and four qubit system. The qubit systems use superconducting charge qubits known as \textit{transmons}. Transmons are used because of the reduced sensitivity to charge
noise to aid in reducing error. The states zero and one are represented as energy states of the transmon, $j$. Each transmon location in the system is represented as $k$ that receive pulses from the error correction circuit over a given amount of time, $t$. The shift in frequencies sent to each transmon is represented as $\Delta_k(t)$ (bounded between $-2.5$ and $2.5$ MHz) and anharmonicity of the shift in frequency of each transmon is represented as $\eta_{jk}$ that is measured to be $200$ MHz for the circuit. The
energy of each $k^{th}$ transmon at each energy level $j$ is
\begin{equation}
    \label{eq:energy}
    E_{kj} = h(j\Delta_k(t)-\eta_{jk}),
\end{equation}
where $h$ is planks constant. As mentioned earlier qubits can entangle with one another, this interaction is represented as a nearest-neighbour coupling strength, $g_k$, between each $kth$ and $(k+1)th$ transmon. The coupling strength is set as $30$ MHz in the simulation.  

The energy transition between each transmon is represented as a $j^n$ block-diagonal Hamiltonian matrix
\begin{equation}
  \label{eq:hamiltonian}
  \frac{\hat{H}\big(\delta_k(t)\big)}{h} = \sum^n_{k=1} \begin{pmatrix} 0&0& 0&0\\0&\varepsilon_k(t)&0&0\\0&0&2\varepsilon_k(t)-\eta&0\\0&0&0&3\varepsilon_k(t)-\eta'\end{pmatrix}_{k}  + \sum^{n-1}_{k=1} \frac{g_k}{2}(X_kX_{k+1}+Y_{k}Y_{k+1}),  
\end{equation}
where each block corresponds to fixed number of excitations. The $X$ and $Y$ are the coupling operators \cite{Ghosh2013}
\begin{align}
    \label{eq: coupling operators}
       X_k=\begin{pmatrix} 0&1& 0&0\\1&0&\sqrt{2}&0\\0&\sqrt{2}&0&\sqrt{3}\\0&0&\sqrt{3}&0\end{pmatrix}_{k}, \quad
  \frac{Y_k}{\text i}
  =\begin{pmatrix} 0&-1& 0&0\\1&0&-\sqrt{2}&0\\0&\sqrt{2}&0&-\sqrt{3}\\0&0&\sqrt{3}&0\end{pmatrix}_{k}, 
\end{align}
are the generalized Pauli operators.
By knowing the Hamiltonian \eqref{eq:hamiltonian} of the transmons system the evolution operator of the system over a time $t$ is represented as
\begin{equation}
  \label{eq:evolution operator}
  U\big( \Delta_k(\Theta) \big) = T e^{\Big\{ -i \int_{0}^{\Theta} \hat{H}\big( \Delta_k(t) \big) dt \Big\} }, 
\end{equation}
Where $\Theta$ is time duration of the error correction and $T$ is the time ordered evolution operator \cite{}. 

Because transmons states have to fall between zero or one (bias states) when observed, high energy levels are not considered when correcting the error. To do this a projection is taken on the evolution operator \eqref{eq:evolution operator} to obtain the computation subspace
\begin{equation}
    \label{eq:projected}
    U_{\mathscr{P}} \big(\Delta_k (\Theta) \big) =  \mathscr{P} U \big(\Delta_k(\Theta) \big) \mathscr{P}.
\end{equation}

The computational subspace \eqref{eq:projected} is the performance metric intrinsic fidelity. To simplify this down to a single percentage value, the percentage of the intrinsic fidelity is represented as 
\begin{equation}
    \mathscr{F}\big(\Delta_k(\Theta)\big)=\frac{1}{N}\Bigg| \mathrm{Tr}\bigg( CCCNot^{\dagger} U_\mathscr{P}\big(\Delta_k(\Theta) \big) \bigg) \Bigg|,
\end{equation}
where CCCNot is the ideal gate \cite{}. 

This model represent the objective function to optimize the frequency shifts, $\Delta_k(t)$, for a $n$-transmons system to obtain feasible intrinsic fidelity of $99.99\%$ for a minimal duration time $\Theta$.

The optimization problem is presented as follows
\begin{equation}
    \label{eq:feasibility}
    0.9999 \leq f(x),
\end{equation}
where a feasible solution is just needed. 

To solve this problem for the four-qubit ($n=4$) and three-qubit ($n=3$) case we used \textit{MATLABs} global search algorithm from the global optimization toolbox with a non-linear constraint to represent the feasibility condition \eqref{eq:feasibility}. To minimize the duration time of the simulation a brute force method is used by simply solving each case with a duration time and lowering it to find the minimal value. The Optimization Database is used in the optimization process to
monitor multiple duration time cases and the progress of global optimization. By using this method to solve the problem the following results for the four-qubit case are shown in Table~\ref{tbl:four qubit} and the three qubit case result shown in Table~\ref{tbl:three qubit}. 



The feasible pulse for minimal duration time of $65$ nanoseconds for the four qubit case is shown in Figure~\ref{fig:four qubit}.


The feasible pulse for minimal duration time of $23$ nanoseconds for the three qubit case is shown in Figure~\ref{fig:three qubit}.


By obtaining these results, error correction circuits using the CCCNot gate are developed for further experimental optimization to account for decoherence error to be used in quantum computers.
